\thispagestyle{empty}

\begin{center}
{\large\bfseries Computación y optimización en la nube \\ Implementación de una aplicación de datos abiertos en la nube }\\
\end{center}
\begin{center}
Pablo Jiménez Jiménez \\
\end{center}

%\vspace{0.7cm}

\vspace{0.5cm}
\noindent{\textbf{Palabras clave}: \textit{software libre, open source, API, OpenAPI, GraphQL, pandas, prophet, datos abiertos, DDD, cloud architecture.}
\vspace{0.7cm}

\noindent{\textbf{Resumen}\\
\vspace{0.7cm}
\\
En este proyecto se desarrolla un sistema de información que permitirá a usuarios con
ciertos conocimientos informáticos consultar la evolución sobre las causas de muerte en
España desde el año 1980. Distintos organismos gubernamentales están recopilando esta
información pero resulta imposible poder consultar los datos de forma eficiente y por
tanto poder realizar estudios de cualquier índole que ayuden a mejorar la prevención del sistema sanitario.  
\vskip 0.2in

A lo largo de este documento se detallará como se ha dado solución a este problema
realizando un desarrollo ágil guiado por las historias de los usuarios. Se documentará la
solución ofrecida y la justificación del diseño realizado. Como resultado del proyecto
tendremos unas interfaces de comunicación agnósticas (independientes del lenguaje de
programación) con la que los usuarios podrán obtener los datos que sean de su interés,
posibilitando el filtrado de estos en base a sus variables. Además, se implementan algunas
 operaciones de predicción y generación de gráficos.


\clearpage

\begin{center}
	{\large\bfseries Cloud computing and optimization \\ Implementation of an open data application in the cloud }\\
	
\end{center}
\begin{center}
	Pablo Jiménez Jiménez \\
\end{center}
\vspace{0.5cm}
\noindent{\textbf{Key words}: \textit{open source, API, OpenAPI, GraphQL, pandas, prophet, open data, DDD, cloud architecture.}
\vspace{0.7cm}

\noindent{\textbf{Abstract}\\
\vspace{0.7cm}
\\  
This project provides an information system that will allow users with certain computer skills
to consult the evolution of the causes of death in Spain since 1980. Different government
agencies  are compiling this information but it is impossible to consult the data
in an efficient way and therefore to carry out any kind of studies in order to improve the
prevention of the health system.  
\vskip 0.2in

Throughout this document, it will be detailed how a solution has been found by carrying
out an agile development guided by the described user stories. The solution offered and
the justification for the design are documented. As a result of the project, we will have
an agnostic communication interface (independent of the programming language) with which
users will be able to obtain the data they are interested in, making it possible to filter
them based on their variables. In addition, some prediction and graphic generation
operations are implemented.

\cleardoublepage

\thispagestyle{empty}

\noindent\rule[-1ex]{\textwidth}{2pt}\\[4.5ex]

D. \textbf{Juan Julián Merelo Guervós}, Profesor(a) del departamento de Arquitectura y Tecnología de Computadores.

\vspace{0.5cm}

\textbf{Informo:}

\vspace{0.5cm}

Que el presente trabajo, titulado \textit{\textbf{Computación y optimización en la nube:
Implementación de una aplicación de datos abiertos en la nube}}, ha sido realizado bajo mi
supervisión por \textbf{Pablo Jiménez Jiménez}, y autorizo la defensa de dicho trabajo
ante el tribunal que corresponda.

\vspace{0.5cm}

Y para que conste, expiden y firman el presente informe en Granada a 8 de julio de 2022.

\vspace{1cm}

\textbf{El director: }

\vspace{5cm}

\noindent \textbf{Juan Julián Merelo Guervós}

\chapter*{Agradecimientos}

A mi familia por estar ahí. 

A mis padres por inculcarme desde pequeño la importancia de formarme y por haberlo hecho posible.