\chapter{Conclusiones y trabajos futuros}
Al principio de este documento se definieron unos objetivos a alcanzar, vamos a analizar
en este capítulo si se han cumplido y cuales serian las lineas de trabajo futuras de
acuerdo con las necesidades expresadas por nuestros usuarios.

El principal reto que tenía que satisfacer este trabajo era la exposición de estos datos
de dominio público para que los usuarios diana fueran capaces de poder conocer la
evolución de las causas de muerte. Esto se ha conseguido implementando una serie de
interfaces agnósticas que permiten acceder al sistema desplegado y realizar consultas sin
necesidad de ningún tipo de configuración. 

Esto avanza en la necesidad de mejora que tiene el \Gls{SNS} para prevenir el máximo número
posible de muertes gracias a las aplicaciones integrales que pueden ser desarrolladas
utilizando este sistema.

Me siento orgulloso de haber sido capaz de realizar con éxito un proyecto de ingeniería de
software desde sus inicios, objetivos, planificación de recursos hasta estructurar una
metodología basada en desarrollo ágil que me ha permitido aportar valor inmediato en las
iteraciones y todo lo que he aprendido de ellos son cosas que pasan desapercibidas en otras situaciones.

\section{Trabajos futuros}
Voy a exponer la principal línea de trabajo por la que sería -bajo mi opinión- interesante seguir:
\begin{itemize}
    \item Permitir la integración de distintas fuentes de datos con otros modelos de datos
    también distintos. Por ejemplo, con otras escalas temporales.
\end{itemize}
Este proyecto es libre y está publicado con licencia \cite{gplv3} por lo que, en
correlación también con el desarrollo ágil, cualquier usuario puede abrir tareas y
proponer cambios.