\chapter{Desarrollo de la solución}
A la hora de empezar con la Implementación de la solución, se debe tener en cuenta los requisitos del sistema de acuerdo con los usuarios diana.

\section{Diseño de la solución}
Para el diseño del software se ha utilizado el diseño dirigido por el dominio, DDD \textit{Domain Driven Design}. Este tipo de arquitectura introducida por Eric Evans \cite{ddd_book}: "Domain-Driven-Design - Tackling Complexity in the Hearth of Software, 2004" nos permite organizar el código de forma separada y organizada lo que nos permite poder realizar TDD correctamente por tener las distintas capas separadas y débilmente acopladas unas de las otras. La lógica de negocio reside en un único sitio y lo más cerca del dominio posible lo que nos permite cumplir los principios SOLID fácilmente y conseguimos un código abierto a su extensión pero cerrado a su modificación. 

\subsection{DDD}
En este tipo de arquitectura podemos diferenciar 3 elementos.

\subsubsection{Entidad}
Las entidades son aquellas clases que representan una entidad del dominio. Estas clases deben tener una estructura de datos que represente la información de la entidad y unas responsabilidades concretas. Se caracterizan porque tienen que ser consideradas iguales a otros objetos aún cuando cuando sus atributos difieren.

\subsubsection{Value objects}
Representan los conceptos que no tienen una entidad. Describen características por lo que solo nos interesan sus atributos.
Los value object representan elementos del modelo que se describen por el \textbf{qué} son, y no por \textbf{quién} o \textbf{cuál} son.

\subsubsection{Servicios}
No tienen significado propio suponen la capa que representa las operaciones que no pertenecen conceptualmente a ningún objeto del dominio concreto.


\section{Implementación}
\subsection{Lenguaje de programación}
El lenguaje elegido ha sido \textbf{Python} los motivos de esta decisión son variados, por los que procedo a enumerarlos:
\begin{itemize}
    \item Es un lenguaje que se encuentra en las \cite{tiobe} primeras posiciones como lenguaje más utilizado a día de hoy. Además me encuentro muy cómodo programando en él.
    \item Es un lenguaje que cuenta con muchísimas bibliotecas, muchas de estas especiales para trabajar con datos y ficheros (como \textbf{Pandas}). Abundan también los \textit{frameworks} que nos ayudan a construir APIs con él y la mayoría de PaaS \footnote{Platform as a Service} soportan este lenguaje.
    \item Es un lenguaje interpretado, débilmente tipado y con una capacidad introspectiva muy alta que nos permite metaprogramar.
\end{itemize}

A la hora de realizar un proyecto software siempre es necesario utilizar bibliotecas y código implementado por otras personas para poder llegar más lejos. Todo esto es lo que llamamos dependencias. Muchos entornos de programación tienen gestores de dependencias que nos permiten instalar y gestionar estas dependencias cómodamente. En el entorno JavaScript tenemos \textit{npm} o \textit{yarn} que nos permite instalar dependencias de forma sencilla. Para la JVM el tradicional \textit{maven} es una buena opción o la mas moderna y potente \textit{Gradle}.

En Python se ha extendido el uso de \textbf{Poetry} que de forma muy parecida a \textit{npm} genera un fichero de configuración \textit{pyproject.toml} donde definimos las dependencias, las versiones y los paquetes de nuestro proyecto así como algunas tareas específicas para correr los tests o el lint fácilmente. Esto facilita enormemente la creación de entornos virtuales de prueba y de desarrollo tanto el local como por el CI.

\subsection{Bibliotecas utilizadas}
\subsubsection{Pandas}
Como se ha descrito en anteriores capítulos de la memoria, los datos que manejamos están en ficheros CSV. Para poder trabajar con ellos es necesario utilizar una biblioteca que nos permite trabajar con estos ficheros. Es necesario poder manejar los datos de estos ficheros con cierta agilidad. El mayor fichero tiene 967360 líneas y ocupa unos 55 MB aunque no son cifras cercanas a lo que se considera \textit{big data} necesitamos de una herramienta que nos permita filtrar, ordenar y hacer cálculos eficientemente sobre estos conjuntos de datos. La elección de \textbf{Pandas} como biblioteca para trabajar con estos datos es debido a que es ampliamente utilizada en el mundo Python y encaja en los datos que tenemos.

\section{Continuos Integration}
Desde el inicio del trabajo se ha estado trabajando con un sistema férreo de CI de forma que no se ha mezclado nunca nada que no haya superado los requisitos de diseño especificados. En el entorno de desarrollo se ha utilizado \textit{Github Actions} que es una herramienta de CI que nos permite realizar una integración continua de la misma forma que \textit{Travis CI} la elección de Github sobre Travis es por la rapidez del elegido e integración con la forja utilizada que como se ha especificado anteriormente es Github.

Para garantizar la \textbf{calidad de la memoria} se han implementado dos tipos de CI\footnote{Continuos Integration}: un revisor ortográfico y un revisor de estructura que compila el documento automáticamente y pública el PDF generado en las \href{https://github.com/pablojjimenez/TFG/releases}{\textit{releases} del repositorio.}

Del mismo modo, para el \textbf{código} se han implementado otras dos CI que garantizan que cuando se introduce en la rama principal del repositorio código nuevo, éste garantiza unos estándares de calidad y que efectivamente el código es funcional por todas las versiones especificadas de Python. Por un lado, verificar el código fuente contra errores de programación (como "library imported but unused" y "Undefined name") y para verificar la complejidad ciclomática. Se ha utilizado la tecnología \cite{flake}\textbf{Flake8} por ser las más utilizada en el ecosistema.
Por otro lado, se corre la \textit{suit} de tests de forma que se garantice que al añadir uno código no se rompe el existente. Se ha utilizado la librería \textit{Unittest} que se encuentra incluida en la biblioteca estándar de Python permite realizar aserciones, provee de muchas aserciones y permite organizar los tests de forma parecida a como lo hace jUnit en clases que llamamos \textit{suits} de tests.

Todas estas comprobaciones se realizan para todas las versiones de Python en las que garantizamos que el software funciona.