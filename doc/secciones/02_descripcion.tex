\chapter{Descripción del problema}

Este Trabajo Final de Grado aborda el problema de la obtención de información de datos 
abiertos publicados en ficheros en crudo sin ningún tipo de organización ni depuración.

En cumplimiento de la Ley 37/2007 de 16 de noviembre y Real Decreto 1495/2011 de 24 de 
octubre sobre la reutilización de la información del sector público por la cual se regula 
la reutilización de los documentos elaborados o custodiados por organismos del sector 
público del Estado español que son publicados para incrementar y reforzar la 
transparencia en la actividad pública, se publicaron diversas páginas como: https://datos.gob.es/ 
y otras propias a cada Comunidad Autónoma donde se publican ingestantes cantidades de 
datos a disposición de la ciudadanía. El problema reside en lo poco útiles que son
estos datos por si solos. 

Para poder obtener riqueza de estos datos es necesario crear lo que 
llamamos un SSOT, \textit{Single Source of Truth}. Es un concepto cuya estrategia radica 
en la creación de un único punto de encuentro fáctico para todas las decisiones a 
realizar sobre una disciplina concreta y se construye a partir de un o varios contenedores 
de datos interesados.

Tomar decisiones a partir de un SSOT nos permite tomar acciones más inteligentes 
y de mayor acierto. Esto se consigue mediante la eliminación de la reduciendo la 
redundancia y por tanto reduciendo imprecisiones en la toma de decisiones lo que 
conlleva directamente a una reducción de esfuerzo comprensivo y analítico incrementando 
las capacidades de análisis inteligente. 

Proporcionar de una herramienta que actúe como SSOT para tomar decisiones
más afortunadas a la hora de rastrear y prevenir causas de muerte entre la 
población es el objetivo que persigo mediante el presente trabajo. En futuros párrafos iré
desentrañando la elaboración.