\chapter{Análisis de los datos disponibles}

Los datos acerca de las defunciones son un componente fundamental en este trabajo sin
el cual no podríamos construir nuestra solución. He necesitado obtener los datos recolectados
durante años por distintas administraciones para usarlos.

En primer lugar, me dirigí a la página sobre la iniciativa de datos abiertos del 
Gobierno de España https://datos.gob.es/ la información que encontré relativa a las defunciones
según causa de muerte son unas colecciones que recolecta anualmente el Instituto
Nacional de Estadística. El formato disponible de descarga era CSV y las columnas de información
propocionadas eran escasas, además solo ofrecen el número crudo de defunciones, dato que nos dificulta 
a la hora de comparar las causas con otra comunidad autonóma pues es necesari tener en cuenta el valor significativo
de la cifra sobre el número de habitantes. Además la clasificación de las enfermedades es muy genérica.

Tras conocer la existencia de los servidores Arïadna y Raziël me puse en contacto con el Insituto de Salud Carlos
III para obtener los datos de dominio público almacenado en esos servidores. En esta ocasión estas colecciones
de datos si son completas, nos ofrecen datos normalizados 

Las principales diferencias entre Arïadna y Raziël es que Arïadna no guarda informació acerca
de los grupos de edades, solo nos ofrece el cómputo de la edad media, cómputo que conociendo
los rangos de edades pdemos calcular nosotros. 
Raziël no almacena la provincia que es algo que si hace Arïadna, me parece interesante guardar al menos
los datos cuantificados ordenados por provincias por que puede ser útil para construir algunas soluciones.
Finalmente también queremos añadir un campo a las causas de Raziël para almacenar la clasificación de las 
enfermedades según el estándar CIE-10 que es algo que si posee Arïadna.

\section{Modelización de los datos}
Vamos a distinguir las siguientes entidades: