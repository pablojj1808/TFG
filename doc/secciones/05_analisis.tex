\chapter{Análisis de los datos disponibles}

Los datos acerca de las defunciones son un componente fundamental en este trabajo sin
el cual no podríamos construir nuestra solución. He necesitado obtener los datos recolectados
durante años por distintas administraciones para usarlos.

En primer lugar, me dirigí a la página sobre la iniciativa de \href{https://datos.gob.es/}{datos abiertos del Gobierno de España}
la información que encontré relativa a las defunciones
según causa de muerte son unas colecciones que recolecta anualmente el Instituto
Nacional de Estadística. El formato disponible de descarga era CSV y las columnas de información
proporcionadas eran escasas, además solo ofrecían el número crudo de defunciones. Nos interesa que existieran
otras variables estadísticas como el número de defunciones por cada 100 000 habitantes. Además la clasificación de
las enfermedades es escasa y genérica.

Tras conocer la existencia de los servidores Arïadna y Raziël me puse en contacto con el \textbf{Insituto de Salud Carlos
III} para obtener los datos de dominio público almacenado en esos servidores. Esta colección de datos si es muy completa
nos ofrece muchos tipos de enfermedades y nos ofrece datos normalizados: la tasa por cada 100 000 habitantes, tasas truncadas y
tasas ajustadas a la población europea entre otras.

Las principales diferencias entre \textbf{Arïadna} y \textbf{Raziël} es que Arïadna no guarda informació acerca
de los grupos de edades, solo nos ofrece el cómputo de la edad media. Este cómputo podemos calcularlo nosotros conociendo
los rangos de edad. Por tanto descarté Arïadna.
Finalmente también queremos añadir un campo a las causas de Raziël para almacenar la clasificación de las 
enfermedades según el estándar \gls{CIE-10} que es algo que si posee Arïadna.

\section{Modelización de los datos}
Nuestro conjunto de datos queda modelizado bajo la entidad \textbf{Raziel} que es la que almacena todas las variables
disponibles por las que filtrar. El resto de entidades que componen el dominio del problema son: grupos de edad,
Comunidad Autónoma, Enfermedad.
