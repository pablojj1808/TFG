\chapter{Análisis de los datos}

Al comenzar el estudio el primer sitio al que fui para obtener este tipo de datos fue el \cite{INE} INE, en su clasificación
\textbf{Defunciones según la Causa de Muerte} el fichero CSV ofrecido por este es escaso en contenido, solo tenemos
número cruso de defunciones por capitulos de causas. En primeras instancias este descubriendo dificultó la realización
de este trabajo porque como es obvio, sin los datos es imposible poder ofrecer una interfaz unificada de consulta.

Tras descubrir la exitencia de los servicios del \cite{isciii} Instituto de Salud Carlos III me puse en contacto
con sus servicios informáticos para conocer si podría tener acceso a los datos almacenados en sus distintos servidores
al ser estos datos de dominio público no tuve mayor problema y en una semana disponía de esta información.

Fue entonces cuando pude comprobar que efectivamente los datos del Servidor Arïadna ofrecía unas variables distintas
a los del Servidor Raziël el próposito principal del trabajo era poner estos datos accesibles y unificados a la población
así que la tarea de fusión era inevitable. 
Para ello se ha hecho uso de la herramienta Pandas
(En el próimo cápitulo se comentará en detalle el porque de la elección 
de esta tecnología.)
