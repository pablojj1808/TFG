\chapter{Elección de la plataforma de gestión documental}

Esta es una de las elecciones más significativas debido a que toda la gestión documental
y la manera en la que se realiza depende del software que utilicemos.
\\

En este caso en particular vamos a utilizar \textbf{algo} debido a que se adapta mejor a mis
necesidades que mi segunda opción, \textbf{algo}.
\\

Los motivos por los que he elegido  frente a Alfresco son los siguientes:

\begin{itemize}
	\item Alfresco sigue la filosofía \textit{Open Source} como método de trabajo, sino que la utilizan a modo de marketing
	 			ya que podemos comprobar que en su desarrollo solo intervienen trabajadores de dicha empresa a pesar de ser bastante
				más grande que .
	\item Alfresco tiene dos versiones, \textit{Community edition} y \textit{Commercial edition}. Esto no sería un problema
				si en la versión de la comunidad se incluyera \textbf{toda la funcionalidad}. Pero de esta manera, si quieres tener
				el software al  deberás pagar la versión comercial. Sin embargo,  nos proporciona toda la funcionalidad
				base tengas o no tengas  una suscripción.
	\item La tecnología que impulsa  es también de código libre y actual.
	\item La \textit{API} de  es mucho más completa y sencilla, así como su documentación.
\end{itemize}
